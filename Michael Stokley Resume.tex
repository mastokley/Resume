%% start of file `template.tex'.
%% Copyright 2006-2013 Xavier Danaux (xdanaux@gmail.com).
% 
% This work may be distributed and/or modified under the
% conditions of the LaTeX Project Public License version 1.3c,
% available at http://www.latex-project.org/lppl/.


\documentclass[10pt,a4paper,sans]{moderncv}        % possible options include font size ('10pt', '11pt' and '12pt'), paper size ('a4paper', 'letterpaper', 'a5paper', 'legalpaper', 'executivepaper' and 'landscape') and font family ('sans' and 'roman')

% moderncv themes
\moderncvstyle{banking}                             % style options are 'casual' (default), 'classic', 'oldstyle' and 'banking'
\moderncvcolor{blue}                               % color options 'blue' (default), 'orange', 'green', 'red', 'purple', 'grey' and 'black'
% \renewcommand{\familydefault}{\sfdefault}         % to set the default font; use '\sfdefault' for the default sans serif font, '\rmdefault' for the default roman one, or any tex font name
% \nopagenumbers{}                                  % uncomment to suppress automatic page numbering for CVs longer than one page

% character encoding
\usepackage[utf8]{inputenc}                       % if you are not using xelatex ou lualatex, replace by the encoding you are using

\input{glyphtounicode}                             % for export
\pdfgentounicode=1

\usepackage{verbatim}

% adjust the page margins
\usepackage[scale=0.95]{geometry}
\setlength{\hintscolumnwidth}{3cm}                % if you want to change the width of the column with the dates
% \setlength{\makecvtitlenamewidth}{10cm}           % for the 'classic' style, if you want to force the width allocated to your name and avoid line breaks. be careful though, the length is normally calculated to avoid any overlap with your personal info; use this at your own typographical risks...

% personal data
\name{Michael}{Stokley}
% \title{Resumé title}                               % optional, remove / comment the line if not wanted
\address{Seattle, WA}%{postcode city}{country}% optional, remove / comment the line if not wanted; the "postcode city" and "country" arguments can be omitted or provided empty
\phone[mobile]{(303) 801-8575}                   % optional, remove / comment the line if not wanted; the optional "type" of the phone can be "mobile" (default), "fixed" or "fax"
% \phone[fixed]{+2~(345)~678~901}
% \phone[fax]{+3~(456)~789~012}
\email{michael@michaelstokley.com}                               % optional, remove / comment the line if not wanted
\homepage{michaelstokley.com}                         % optional, remove / comment the line if not wanted
\social[linkedin]{mastokley}                        % optional, remove / comment the line if not wanted
% \social[twitter]{jdoe}                             % optional, remove / comment the line if not wanted
\social[github]{mastokley}                              % optional, remove / comment the line if not wanted
% \extrainfo{additional information}                 % optional, remove / comment the line if not wanted
% \photo[64pt][0.4pt]{picture}                       % optional, remove / comment the line if not wanted; '64pt' is the height the picture must be resized to, 0.4pt is the thickness of the frame around it (put it to 0pt for no frame) and 'picture' is the name of the picture file

% ----------------------------------------------------------------------------------
% content
% ----------------------------------------------------------------------------------
\begin{document}
% -----       resume       ---------------------------------------------------------
\makecvtitle

\section{Profile}
Software developer. I'm patient enough to take difficult projects seriously, and
driven enough to automate tedious, needlessly manual work. In my previous
accounting position, I exceeded expectations by learning SQL to automate my data
entry / data quality tasks (and avoid using a cumbersome GUI). Since then, I
have spent two years studying math, computer science, and statistics at North
Seattle College, attended Code Fellows' Python Development Intensive (taught by
esteemed pythonista Cris Ewing). I've been working as a foundational member of
the engineering team at iDatalabs, building out a customer facing web portal
with Vue.js, Flask, and PostgreSQL. My goals for 2018: learn more diverse
languages (e. g. functional, typed, low-level, or systems); learn to handle
asynchrony without callbacks (using, for example, channels found in Go or
Clojure); blog about design patterns; work through a few more chapters of SICP.

\section{Technical Skills}
\begin{cvcolumns}
  \cvcolumn{Proficient}{
    Python,
    Django,
    % Django REST Framework,
    Flask,
    PostgreSQL,
    JavaScript/HTML/CSS,
    Vue.js,
    Git,
    SQL,
    Unix,
    % Bash,
    MVC,
    HTTP,
    OOP,
    FP,
    Highcharts
  }
  \cvcolumn{Familiar}{
    Webpack,
    AWS Deployment,
    Heroku Deployment,
    Ansible,
    Nginx,
    Gunicorn,
    MongoDB,
    Pyramid,
    Scheme
  }
\end{cvcolumns}

\begin{comment}
  \section{Technical Skills}
  \subsection{Proficient}
  Python, Django, Django REST Framework, Pyramid, Git, SQL, HTML/CSS,
  JavaScript/jQuery, Unix systems, Shell Scripting, Org-mode, Regular
  Expressions
  \subsection{Familiar}
  AWS Deployment, Heroku Deployment, Ansible, PostgreSQL, Nginx, Gunicorn,
  Scheme, Sed
\end{comment}

\section{Experience}

\begin{comment}
  \cventry
  {date - date}
  {Employer}
  {Job Title}
  {}
  {City, State}
  {Description of work
    \begin{itemize}
    \item specific achievement
    \item specific achievement
    \end{itemize}}
\end{comment}

\cventry
{2017.01 - present}
{iDatalabs}
{Full Stack Engineer}
{}
{Bellevue, WA}
{I built and currently maintain a Vue.js single page application that displays and sells
  marketing data; I also maintain and update the consumed api, written in Flask
  (Python) as well as migrate underlying db schema to fit api requirements as
  needed. I work on and debug all tiers of the architecture, from UI to backend,
  to our Postgres and Mongo databases, to the various Apache and Nginx servers
  deployed across Azure instances.
  \begin{itemize}
  \item Designed and implemented a 'store' to manage client-side shared state
  \item Optimized perceived page performance, including both frontend and
    backend optimizations
  \item Knocked up internal admin dashboards in Flask as needed
  \end{itemize}}

\cventry {2016.08 - 2016.10} {Red Mountain Software} {Freelance Web Developer}
{} {Seattle, WA}
{Designed, built, and tested an internal management application
  according to client specification. Carefully designed and implemented database
  schema to fit domain-specific business objects. App built with Django and PostgreSQL, with
  jQuery on the front as needed.}

\cventry {2016.07 - 2016.09} {Sternshus, Inc} {Co-founding Software Developer}
{}
{Seattle, WA}
{Provided front-end solutions in the context of rapidly
  changing business requirements. Worked primarily in Webflow, Django, and
  MySQL. I am most proud of having written sed and python scripts to automate
  merging Webflow files into the larger Django project.
}

\cventry {2012.04 - 2015.04} {Cleanscapes} {AR Specialist} {} {Seattle, WA}
{Wrote and maintained a library of accurate, sophisticated SQL queries to
  capture evolving data inconsistencies; these queries eliminated 37.5 labor
  hours per month despite a 126\% increase in customer accounts over a two year
  period
}

\section{Projects}

\begin{comment}

  \cventry
  {date}
  {github url}
  {name}
  {}
  {}
  {description}

\end{comment}

\cventry
{2016.10}
{\href{http://michaelstokley.com/Projects/nord-entities}{michaelstokley.com/Projects/nord-entities}}
{Nord Entities}
{}
{}
{Literate programming exploration of natural language processing techniques such as
  tokenizing, chunking, and named entity recognition. Written in Python with
  Requests, the NLTK, and Beautiful Soup.
}

\cventry
{2016.06}
{\href{http://michaelstokley.com/Projects/twitter-haiku}{michaelstokley.com/Projects/twitter-haiku}}
{Twitter Haiku}
{}
{}
{Literate programming exercise in web scraping, tokenizing, and functional
  programming techniques learned from the SICP. Written in Python.
}

\cventry {2016.04}
{\href{http://michaelstokley.com/Projects/toy-domain-scraper}{michaelstokley.com/Projects/toy-domain-scraper}}
{Toy Domain Scraper} {} {} {Toy web scraper built in Python. Uses breadth-first
  traversal for fast, reliable results.
}

\begin{comment}
  \cventry {2016.04}
  {\href{http://github.com/TeamReciprocity/reciprocity}{github.com/TeamReciprocity/reciprocity}}
  {Reciprocity} {} {} {A full stack Python web application for recipe sharing and
    collaboration. Built with Django and PostgreSQL, deployed via Ansible to an
    AWS EC2 instance.
  }
\end{comment}

\begin{comment}
  \cventry {2016.04}
  {\href{https://github.com/homequest/homequest}{github.com/homequest/homequest}}
  {HomeQuest} {} {} {A full-stack Python web application to gamify household
    chores and chore management. Manually implemented authorization,
    authentication, and row-level security. Built in Pyramid with a PostgreSQL
    database and deployed on Heroku.
  }
\end{comment}

\section{Education}
\cventry {2016} {Certificate} {Code Fellows} {Seattle, WA} {} {Advanced Python development emphasizing test driven development, pair programming, agile methodologies, version control, and basic data structures and algorithms}

\cventry {2012--2014} {Coursework in Math, Computer Science, and Statistics}
{North Seattle College} {Seattle, WA} {} {}

\cventry {2009} {B. A. in Philosophy} {St. John's College}
{Annapolis, MD} {} {}
%{Double Major: Philosophy and the History of Math and Science\newline{} Double Minor: Classical Studies and Comparative Literature}

\end{document}
